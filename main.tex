\documentclass[10pt]{beamer}
\usepackage[utf8]{inputenc}
\usepackage[T1]{fontenc}
\usepackage[french]{babel}
\usepackage{listings}
\usepackage{bera} 

\usepackage{tikz}

\usepackage{hyperref}
\newcommand{\sourceline}[1]{%
  \begin{tikzpicture}[remember picture,overlay]
    \node[anchor=south east, xshift=-0.2cm, yshift=0.5cm] at (current page.south east) {%
      \scriptsize\textit{Source:}~\href{#1}{\texttt{#1}}};
  \end{tikzpicture}%
}


% Thème Beamer
\usetheme{Madrid}


\title{Test par Fuzzing}
\author{Yamir Alejandro Poldo Silva}
\date{2 Octobre 2025}

\begin{document}

% Page de titre
\begin{frame}
  \titlepage
\end{frame}

\begin{frame}{Plan de la présentation}
\tableofcontents
\end{frame}

% Introduction
\section{Introduction}
\begin{frame}{Introduction}
\begin{itemize}
  \item Les tests unitaires sont essentiels mais limités.
  \item Même avec de bons tests unitaires, des failles critiques peuvent exister.
  \item Le \textbf{fuzzing} comble ce vide en injectant massivement des données.
\end{itemize}
\end{frame}

% Définition
\section{Définition}
\begin{frame}{Définition du fuzzing}
\begin{block}{Définition}
Fuzzing (ou fuzz testing) est une technique de test automatisée qui consiste à fournir massivement \textbf{des entrées aléatoires, corrompues ou spécialement construites} à un programme afin d’observer son comportement et déceler des erreurs \textbf{(crashs, fuites mémoire, exceptions, réponses anormales)}.

\sourceline{https://github.com/resources/articles/security/what-is-fuzz-testing}
\end{block}
\end{frame}



% Génération des données
\section{La génération de donnée}
\begin{frame}{Génération des données}
Deux grandes approches :
\begin{itemize}
    \item \textbf{Génération} : produire des entrées à partir de modèles/grammaires (JSON, protocoles...).
    \item \textbf{Mutation} : modifier une entrée valide (seed).
\end{itemize}
\sourceline{https://github.com/resources/articles/security/what-is-fuzz-testing}
\end{frame}

\begin{frame}[fragile]{Exemple : génération basée sur une grammaire}
\begin{block}{Une grammaire}
\begin{lstlisting}[basicstyle=\ttfamily\scriptsize]
<start>   ::= <expr>
<expr>    ::= <term> + <expr> | <term> - <expr> | <term>
<term>    ::= <term> * <factor> | <term> / <factor> | <factor>
<factor>  ::= +<factor> | -<factor> | (<expr>) | <integer>
<integer> ::= <digit><integer> | <digit>
<digit>   ::= 0|1|2|3|4|5|6|7|8|9
\end{lstlisting}
\end{block}

\begin{itemize}
  \item Exemples générés : \texttt{(1+2)*3}, \texttt{4/(2-1)}, \texttt{12+34*5}, \texttt{-(3+4)}
\end{itemize}

\vspace{1mm}
\sourceline{https://www.fuzzingbook.org/html/Grammars.html}
\end{frame}


% Exemple Radamsa
\begin{frame}[fragile]{Exemple avec Radamsa (mutation)}
\begin{block}{Principe}
Radamsa prend une entrée valide et génère plusieurs variantes corrompues.
\end{block}

\begin{lstlisting}[language=bash, frame=single]
$ echo "1 + (2 + ( 3 + 4))" | radamsa --seed 12 -n 4
1 + (2 + (2 + (3 + 4?))
1 + (2 + (3 + ?4))
18446744073709551615 + 4)))
1 + (2 + (3 + 170141183460469231731687303715884105727))
\end{lstlisting}

\begin{itemize}  
  \item Le fuzzer explore ainsi des cas non anticipés (valeurs extrêmes, caractères inattendus).  
\end{itemize}

\sourceline{https://gitlab.com/akihe/radamsa}

\end{frame}


% Types de fuzzing
\section{Les principales approches}
\begin{frame}{Les principales approches}
\begin{itemize}
  \item \textbf{Boîte noire} : aucune connaissance du fonctionnement interne du programme, génération d'entrée aléatoires.
  \item \textbf{Boîte blanche} : accès au code source, génération d’entrées spécifiques.
  \item \textbf{Boîte grise} : Connaissances partielles du système.
  \item \textbf{Guidée par la couverture} : Génération et priorisation des entrées qui augmente la couverture du code. 
\end{itemize}
\sourceline{https://github.com/resources/articles/security/what-is-fuzz-testing}
\end{frame}


% Cycle de fonctionnement
\section{Cycle de fonctionnement}
\begin{frame}{Cycle de fonctionnement d’un fuzzer}
\begin{enumerate}
  \item Définir les objectifs et la portée. \textbf{Les systèmes testés doivent accepter une entrée}.
  \item Choisir l\’ outil selon les contraintes de tests et langage de programmation.
  \item Générer ou muter des entrées et les injecter dans la cible.
  \item Observer et enregistrer les anomalies \textbf{(plantage, fuites de mémoires, buffer overflows, exceptions, et autres comportements inattendues)}
  \item Corriger les erreurs à partir du rapport généré et relancer la campagne de fuzzing.
\end{enumerate}

\sourceline{https://github.com/resources/articles/security/what-is-fuzz-testing}
\end{frame}

% Outils
\section{Quelques outils de fuzzing}
\begin{frame}{Outils de fuzzing}
\begin{itemize}
  \item \textbf{AFL / AFL++} — fuzzer basé sur la couverture.
  \item \textbf{Jazzer} — fuzzer basé sur la couverture pour la JVM / applications Java et Kotlin.
  \item \textbf{OSS-Fuzz} — service de fuzzing continu (Google).
  \item \textbf{Radamsa} — outil de \emph{mutation} simple.
  \item \textbf{FFUF (Fuzz Faster U Fool)} — fuzzer léger orienté applications web / URL discovery (Go). Laboratoire en ligne \url{ffuf.me}
\end{itemize}
\sourceline{https://github.com/secfigo/Awesome-Fuzzing?tab=readme-ov-file#tools}

\end{frame}

\section{Exemple de Fuzzer minimaliste}
\begin{frame}[fragile]
\frametitle{Exemple : FFUF + Radamsa}

\begin{lstlisting}[language=bash, basicstyle=\small\ttfamily, frame=single]
# Generer 1000 mutations avec Radamsa
$ echo '{"user":"admin","email":"test@example.com"}' | \
  radamsa --seed 42 -n 1000 > payloads.txt

# Fuzzer l'endpoint avec FFUF
$ ffuf -u https://httpbin.org/post \
       -X POST \
       -H "Content-Type: application/json" \
       -d FUZZ \
       -w payloads.txt \
       -mc all \
       -t 10 \
       -maxtime 60
\end{lstlisting}

\end{frame}


% Limites
\section{Bénéfices et limites}
\begin{frame}{Bénéfices et limites}
\begin{itemize}
  \item Découverte de bugs inattendus (même dans du code déjà testé)
  \item Améliore la robustesse et la sécurité logicielle
  \item Complément aux tests unitaires et analyses statiques
  \item Efficace par rapport aux tests manuels et humains. 
\end{itemize}
\pause
\begin{block}{Limites}
\begin{itemize}
  \item Ne couvre pas toutes les voies d’exécution
  \item Bugs parfois difficiles à analyser ou reproduire
  \item Peut être non-déterministe 
  \item Coût élevé en ressources et temps 
\end{itemize}
\end{block}
\sourceline{https://github.com/resources/articles/security/what-is-fuzz-testing}
\end{frame}


% Conclusion 
\section{Conclusion}
\begin{frame}{Conclusion}
\begin{itemize}
  \item Important que bien de connaître le système testé
  \item Complémentaire aux autres tests et aux bonnes pratiques
\end{itemize}
\end{frame}

\begin{frame}{Sources principales}
\footnotesize
\begin{thebibliography}{9}

\bibitem{github}
GitHub Security Lab. 
\newblock \emph{What is fuzz testing?}  
\newblock  \url{https://github.com/resources/articles/security/what-is-fuzz-testing} (consulté le 1 octobre 2025).

\bibitem{fuzzingbook}
Zeller, A. et al.  
\newblock \emph{The Fuzzing Book}.  
\newblock \url{https://www.fuzzingbook.org/} (consulté le 20 septembre 2025).

\bibitem{radamsa}
Helin, A.  
\newblock \emph{Radamsa Fuzzer}.  
\newblock \url{https://gitlab.com/akihe/radamsa} (consulté le 20 septembre 2025).

\bibitem{antithesis}
Antithesis.  
\newblock \emph{SDTalk Blog Post (Mario Bros Example)}.  
\newblock \url{https://antithesis.com/blog/sdtalk/} (consulté le 29 septembre 2025).

\bibitem{ossfuzz}
Google.  
\newblock \emph{OSS-Fuzz}.  
\newblock  \url{https://oss-fuzz.com} (consulté le 29 septembre 2025).

\end{thebibliography}
\end{frame}



\end{document}
